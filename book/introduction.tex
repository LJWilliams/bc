\chapter{Introduction}\label{introduction}

Here's the dream:

\begin{quote}
Computers have revolutionized research, and that revolution is only
beginning. Every day, scientists and engineers all over the world use
them to study things that are too big, too small, too fast, too slow,
too expensive, too dangerous, or just too hard to study any other way.
\end{quote}

Now here's the reality:

\begin{quote}
Every day, scientists and engineers all over the world waste time
wrestling with computers. Tasks that should take a few moments take
hours or days, and many things never work at all. And even when things
\emph{do} work, most scientists have no idea how reliable their results
are.
\end{quote}

Most of the pain that researchers feel stems from not knowing how to
develop software systematically, how to tell if their programs are
working correctly, how to share their work with others (except by
mailing files to one another), or how to keep track of what they've
done. This sorry state of affairs persists for four reasons:

\begin{swcitemize}
\item
  \emph{No room, no time.} Everybody's curriculum is full---there's
  simply not space to add more about computing without dropping
  something else.
\item
  \emph{No standards.} Reviewers and granting agencies don't check
  whether software is correct, ask how long it took to write, or count
  it toward tenure, so there's no incentive for scientists to do better.
\item
  \emph{The blind leading the blind.} Senior researchers can't teach the
  next generation how to do things that they don't know how to do
  themselves.
\item
  \emph{The cult of big iron.} Attention and funding mostly goes to
  things that politicians and university presidents can brag about on
  opening day, rather than to the basic skills that almost everyone
  uses.
\end{swcitemize}

Our goal is to show scientists and engineers how to do more in less time
and with less pain. Our lessons have been used by more than four
thousand learners in over a hundred two-day workshops since the spring
of 2010. Here's how they can help:

\begin{swcitemize}
\item
  If you've ever overwritten the wrong file, we'll show you how to use
  version control.
\item
  If you've ever spent hours typing the same commands over and over
  again, we'll show you how to automate those tasks using simple
  scripts.
\item
  If you've ever spent an afternoon trying to figure out what the
  program you wrote last week actually does, we'll show you how to break
  your code into modules that you can read, debug, and improve piece by
  piece.
\item
  If you've ever spent days copying and pasting data in text files and
  spreadsheets, we'll show you how a database can do the work for you.
\end{swcitemize}

\subsection{About Us}

Software Carpentry is an open source project. Our instructors are
volunteers, and all of our lessons are freely available under the
\urlfoot{http://creativecommons.org/licenses/by/3.0/}{Creative Commons -
Attribution License}, so you can re-use and re-mix them however you want
so long as you cite us as the original source.

Like all volunteer projects, Software Carpentry needs your help to grow.
If you find a bug, please file a report in
\urlfoot{https://github.com/swcarpentry/bc/}{our GitHub repo}. If you would
like to host a workshop, please
get in touch; if you'd like
to teach, we run an
\urlfoot{http://teaching.software-carpentry.org}{instructor training
course}; and if you'd like to write lessons or exercises, please
let us know.

To find out more, please visit the
\urlfoot{Software Carpentry web site}{http://software-carpentry.org}
or read
\urlfoot{http://www.plosbiology.org/article/info\%3Adoi\%2F10.1371\%2Fjournal.pbio.1001745}{these}
\urlfoot{http://arxiv.org/abs/1307.5448}{papers} or
\urlfoot{http://software-carpentry.org/blog/index.html\#popular}{our most
popular blog posts}.

\subsection{Acknowledgments}

Software Carpentry has been made possible by the generous support of:

\begin{swcitemize}
\item
  \urlfoot{http://continuum.io/}{Continuum Analytics}
\item
  \urlfoot{http://www.indiana.edu}{Indiana University}
\item
  \urlfoot{http://www.lbl.gov}{Lawrence Berkeley National Laboratory}
\item
  \urlfoot{http://www.lanl.gov}{Los Alamos National Laboratory}
\item
  \urlfoot{http://www.mathworks.com}{MathWorks}
\item
  \urlfoot{http://www.msu.edu}{Michigan State University}
\item
  \urlfoot{http://www.microsoft.com}{Microsoft}
\item
  \urlfoot{http://www.mitacs.ca}{MITACS}
\item
  \urlfoot{http://mozillafoundation.org}{The Mozilla Foundation}
\item
  \urlfoot{http://www.python.org/psf/}{The Python Software Foundation}
\item
  \urlfoot{http://www.qmul.ac.uk}{Queen Mary University of London}
\item
  \urlfoot{http://www.scimatic.com}{Scimatic Software}
\item
  \urlfoot{http://www.scinet.utoronto.ca}{SciNet}
\item
  \urlfoot{http://www.sharcnet.ca}{SHARCNET}
\item
  \urlfoot{http://www.sloan.org}{The Alfred P. Sloan Foundation}
\item
  \urlfoot{http://www.stsci.edu}{The Space Telescope Science Institute}
\item
  \urlfoot{http://www.metoffice.gov.uk}{The UK Meteorological Office}
\item
  \urlfoot{http://www.ualberta.ca}{The University of Alberta}
\item
  \urlfoot{http://www.ucar.edu}{The University Consortium for Atmospheric Research}
\item
  \urlfoot{http://www.utoronto.ca}{The University of Toronto}
\end{swcitemize}
