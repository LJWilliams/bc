\chapter{Reference}\label{s:reference}

These short reference guides cover the basic tools and ideas introduced
in our lessons.

\section{Shell Reference}

\subsection*{Basic Commands}

\begin{swcitemize}
\item
  \code{cat} displays the contents of its inputs.
\item
  \code{cd path} changes the current working directory.
\item
  \code{cp old new} copies a file.
\item
  \code{find} finds files with specific properties that match
  patterns.
\item
  \code{grep} selects lines in files that match patterns.
\item
  \code{head} displays the first few lines of its input.
\item
  \code{ls path} prints a listing of a specific file or directory;
  \code{ls} on its own lists the current working directory.
\item
  \code{man command} displays the manual page for a given command.
\item
  \code{mkdir path} creates a new directory.
\item
  \code{mv old new} moves (renames) a file or directory.
\item
  \code{pwd} prints the user's current working directory.
\item
  \code{rm path} removes (deletes) a file.
\item
  \code{rmdir path} removes (deletes) an empty directory.
\item
  \code{sort} sorts its inputs.
\item
  \code{tail} displays the last few lines of its input.
\item
  \code{touch path} creates an empty file if it doesn't already exist.
\item
  \code{wc} counts lines, words, and characters in its inputs.
\item
  \code{whoami} shows the user's current identity.
\end{swcitemize}

\subsection*{Paths}

\begin{swcitemize}
\item
  \code{/path/from/root} is an absolute path.
\item
  \code{/} on its own refers to the root of the filesystem.
\item
  \code{path/without/leading/slash} is a relative path.
\item
  \code{.} refers to the current directory, \code{..} to its parent.
\item
  \code{*} matches zero or more characters in a filename, so
  \code{*.txt} matches all files ending in \code{.txt}.
\item
  \code{?} matches any single character in a filename, so
  \code{?.txt} matches \code{a.txt} but not \code{any.txt}.
\end{swcitemize}

\subsection*{Combining Commands}

\begin{swcitemize}
\item
  \code{command \textgreater{} file} redirects a command's output to a
  file.
\item
  \code{first \textbar{} second} connects the output of the first
  command to the input of the second.
\item
  A \code{for} loop repeats commands once for every thing in a list:

\begin{Verbatim}
for variable in name_1 name_2 name_3
do
    ...commands refering to $variable...
done
\end{Verbatim}
\item
  Use \code{\$name} to expand a variable (i.e., get its value).
\item
  \code{history} displays recent commands, and \code{!number} to
  repeat a command by number.
\item
  \code{bash filename} runs commands saved in \code{filename}.
\item
  \code{\$*} refers to all of a shell script's command-line
  parameters.
\item
  \code{\$1}, \code{\$2}, etc., refer to specified command-line
  parameters.
\item
  \code{\$(command)} inserts a command's output in place.
\end{swcitemize}

\section{Git Reference}

Set global configuration (only needs to be done once per machine):

\begin{Verbatim}
git config --global user.name "Your Name"
git config --global user.email "you@some.domain"
git config --global color.ui "auto"
git config --global core.editor "your_editor"
\end{Verbatim}

Initialize the current working directory as a repository:

\begin{Verbatim}
git init
\end{Verbatim}

Display the status of the repository:

\begin{Verbatim}
git status
\end{Verbatim}

Add specific files to the staging area:

\begin{Verbatim}
git add filename_1 filename_2
\end{Verbatim}

Add all modified files in the current directory and its sub-directories
to the staging area:

\begin{Verbatim}
git add -A .
\end{Verbatim}

Commit changes in the staging area to the repository's history: (Without
\code{-m} and a message, this command runs a text editor.)

\begin{Verbatim}
git commit -m "Some message"
\end{Verbatim}

View the history of the repository:

\begin{Verbatim}
git log
\end{Verbatim}

Display differences between the current state of the repository and the
last saved state:

\begin{Verbatim}
git diff
\end{Verbatim}

Display differences between the current state of a particular file and
the last saved state:

\begin{Verbatim}
git diff path/to/file
\end{Verbatim}

Display differences between the last saved state and what's in the
staging area:

\begin{Verbatim}
git diff --staged
\end{Verbatim}

Compare files to the previously saved state:

\begin{Verbatim}
git diff HEAD~1 path/to/file
\end{Verbatim}

Compare files to an earlier saved state:

\begin{Verbatim}
git diff HEAD~27 path/to/file
\end{Verbatim}

Compare files to a specific earlier state:

\begin{Verbatim}
git diff 123456 path/to/file
\end{Verbatim}

Erase changes since the last save:

\begin{Verbatim}
git reset --hard HEAD
\end{Verbatim}

Restore file to its state in a previous revision:

\begin{Verbatim}
git checkout 123456 path/to/file
\end{Verbatim}

Add a remote to a repository:

\begin{Verbatim}
git remote add nickname remote_specification
\end{Verbatim}

Display a repository's remotes:

\begin{Verbatim}
git remote -v
\end{Verbatim}

Push changes from a local repository to a remote (assuming
\code{master} already exists in the remote):

\begin{Verbatim}
git push nickname master
\end{Verbatim}

Push changes from a local repository to a remote (if \code{master}
doesn't yet exist in the remote):

\begin{Verbatim}
git push -u nickname master
\end{Verbatim}

Pull changes from a remote repoisitory:

\begin{Verbatim}
git pull nickname master
\end{Verbatim}

Note: \code{master} may be replaced with another branch name by more
advanced users.

Clone a remote repository:

\begin{Verbatim}
git clone remote_specification
\end{Verbatim}

Markers used to show conflict in a file during a merge:

\begin{Verbatim}
<<<<<<< HEAD
lines from local file
=======
lines from remote file
>>>>>>> 1234567890abcdef1234567890abcdef12345678
\end{Verbatim}

\section{Python Reference}

\subsection*{Basic Operations}

\begin{swcitemize}
\item
  Use \code{variable = value} to assign a value to a variable.
\item
  Use \code{print first, second, third} to display values.
\item
  Python counts from 0, not from 1.
\item
  \code{\#} starts a comment.
\item
  Statements in a block must be indented (usually by four spaces).
\item
  \code{help(thing)} displays help.
\item
  \code{len(thing)} produces the length of a collection.
\item
  \code{{[}value1, value2, value3, ...{]}} creates a list.
\item
  \code{list\_name{[}i{]}} selects the i'th value from a list.
\end{swcitemize}

\subsection*{Control Flow}

\begin{swcitemize}
\item
  Create a \code{for} loop to process elements in a collection one at
  a time:

\begin{Verbatim}
for variable in collection:
    ...body...
\end{Verbatim}
\item
  Create a conditional using \code{if}, \code{elif}, and
  \code{else}:

\begin{Verbatim}
if condition_1:
    ...body...
elif condition_2:
    ...body...
else:
    ...body...
\end{Verbatim}
\item
  Use \code{==} to test for equality.
\item
  \code{X and Y} is only true if both X and Y are true.
\item
  \code{X or Y} is true if either X or Y, or both, are true.
\item
  Use \code{assert condition, message} to check that something is true
  when the program is running.
\end{swcitemize}

\subsection*{Functions}

\begin{swcitemize}
\item
  \code{def name(...params...)} defines a new function.
\item
  \code{def name(param=default)} specifies a default value for a
  parameter.
\item
  Call a function using \code{name(...values...)}.
\end{swcitemize}

\subsection*{Libraries}

\begin{swcitemize}
\item
  Import a library into a program using \code{import libraryname}.
\item
  The \code{sys} library contains:

  \begin{swcitemize2}
  \item
    \code{sys.argv}: the command-line arguments a program was run
    with.
  \item
    \code{sys.stdin}, \code{sys.stdout}: standard input and output.
   \end{swcitemize2}
\item
  \code{glob.glob(pattern)} returns a list of files whose names match
  a pattern.
\end{swcitemize}

\subsection*{Arrays}

\begin{swcitemize}
\item
  \code{import numpy} to load the NumPy library.
\item
  \code{array.shape} gives the shape of an array.
\item
  \code{array{[}x, y{]}} selects a single element from an array.
\item
  \code{low:high} specifies a slice including elements from
  \code{low} to \code{high-1}.
\item
  \code{array.mean()}, \code{array.max()}, and \code{array.min()}
  calculate simple statistics.
\item
  \code{array.mean(axis=0)} calculates statistics across the specified
  axis.
\end{swcitemize}

\section{SQL Reference}

\subsection*{Basic Queries}

Select one or more columns from a table:

\begin{Verbatim}
SELECT column_name_1, column_name_2 FROM table_name;
\end{Verbatim}

Select all columns from a table:

\begin{Verbatim}
SELECT * FROM table_name;
\end{Verbatim}

Get only unique results in a query:

\begin{Verbatim}
SELECT DISTINCT column_name FROM table_name;
\end{Verbatim}

Perform calculations in a query:

\begin{Verbatim}
SELECT column_name_1, ROUND(column_name_2 / 1000.0) FROM table_name;
\end{Verbatim}

Sort results in ascending order:

\begin{Verbatim}
SELECT * FROM table_name ORDER BY column_name_1;
\end{Verbatim}

Sort results in ascending and descending order:

\begin{Verbatim}
SELECT * FROM table_name ORDER BY column_name_1 ASC, column_name_2 DESC;
\end{Verbatim}

\subsection*{Filtering}

Select only data meeting a condition:

\begin{Verbatim}
SELECT * FROM table_name WHERE column_name > 0;
\end{Verbatim}

Select only data meeting a combination of conditions:

\begin{Verbatim}
SELECT * FROM table_name WHERE (column_name_1 >= 1000) AND (column_name_2 = 'A' OR column_name_2 = 'B');
\end{Verbatim}

\subsection*{Missing Data}

Use \code{NULL} to represent missing data.

\code{NULL} is not 0, false, or any other specific value. Operations
involving \code{NULL} produce \code{NULL}, so \code{1+NULL},
\code{2\textgreater{}NULL}, and \code{3=NULL} are all \code{NULL}.

Test whether a value is null:

\begin{Verbatim}
SELECT * FROM table_name WHERE column_name IS NULL;
\end{Verbatim}

Test whether a value is not null:

\begin{Verbatim}
SELECT * FROM table_name WHERE column_name IS NOT NULL;
\end{Verbatim}

\subsection*{Grouping and Aggregation}

Combine values using aggregation functions:

\begin{Verbatim}
SELECT SUM(column_name_1) FROM table_name;
\end{Verbatim}

Combine data into groups and calculate combined values in groups:

\begin{Verbatim}
SELECT column_name_1, SUM(column_name_2), COUNT(*) FROM table_name GROUP BY column_name_1;
\end{Verbatim}

\subsection*{Joins}

Join data from two tables:

\begin{Verbatim}
SELECT * FROM table_name_1 JOIN table_name_2 ON table_name_1.column_name = table_name_2.column_name;
\end{Verbatim}

\subsection*{Writing Queries}

SQL commands must be combined in the following order: \code{SELECT},
\code{FROM}, \code{JOIN}, \code{ON}, \code{WHERE},
\code{GROUP BY}, \code{ORDER BY}.

\subsection*{Creating Tables}

Create tables by specifying column names and types. Include primary and
foreign key relationships and other constraints.

\begin{Verbatim}
CREATE TABLE survey(
    taken   INTEGER NOT NULL,
    person  TEXT,
    quant   REAL NOT NULL,
    PRIMARY KEY(taken, quant),
    FOREIGN KEY(person) REFERENCES person(ident)
);
\end{Verbatim}

\subsection*{Programming}

Execute queries in a general-purpose programming language by:

\begin{swcitemize}
\item
  loading the appropriate library
\item
  creating a connection
\item
  creating a cursor
\item
  repeatedly:

  \begin{swcitemize2}
  \item
    execute a query
  \item
    fetch some or all results
   \end{swcitemize2}
\item
  disposing of the cursor
\item
  closing the connection
\end{swcitemize}

Python example:

\begin{Verbatim}
import sqlite3
connection = sqlite3.connect("database_name")
cursor = connection.cursor()
cursor.execute("...query...")
for r in cursor.fetchall():
    ...process result r...
cursor.close()
connection.close()
\end{Verbatim}
